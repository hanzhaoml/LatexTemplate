%%%%%%%%%%%%%%%%%%%%%%%%%%%%%%%%%%%%%%%%%
% Jacobs Landscape Poster
% LaTeX Template
% Version 1.0 (29/03/13)
%
% Created by:
% Computational Physics and Biophysics Group, Jacobs University
% https://teamwork.jacobs-university.de:8443/confluence/display/CoPandBiG/LaTeX+Poster
% 
% Further modified by:
% Nathaniel Johnston (nathaniel@njohnston.ca)
%
% This template has been downloaded from:
% http://www.LaTeXTemplates.com
%
% License:
% CC BY-NC-SA 3.0 (http://creativecommons.org/licenses/by-nc-sa/3.0/)
%
%%%%%%%%%%%%%%%%%%%%%%%%%%%%%%%%%%%%%%%%%

%----------------------------------------------------------------------------------------
%	PACKAGES AND OTHER DOCUMENT CONFIGURATIONS
%----------------------------------------------------------------------------------------

\documentclass[final,20pt]{beamer}

\usepackage[scale=1.25]{beamerposter} % Use the beamerposter package for laying out the poster
\usepackage{amsmath}
\usepackage[makeroom]{cancel}
\usetheme{confposter} % Use the confposter theme supplied with this template

\setbeamercolor{block title}{fg=CMURed,bg=CMURed!20} % Colors of the block titles
\setbeamercolor{block body}{fg=black,bg=white} % Colors of the body of blocks
\setbeamercolor{block alerted title}{fg=white,bg=dblue!70} % Colors of the highlighted block titles
\setbeamercolor{block alerted body}{fg=black,bg=dblue!10} % Colors of the body of highlighted blocks
\setbeamercolor{itemize item}{fg=CMURed!30} % Colors of the item 
% Many more colors are available for use in beamerthemeconfposter.sty

%-----------------------------------------------------------
% Define the column widths and overall poster size
% To set effective sepwid, onecolwid and twocolwid values, first choose how many columns you want and how much separation you want between columns
% In this template, the separation width chosen is 0.024 of the paper width and a 4-column layout
% onecolwid should therefore be (1-(# of columns+1)*sepwid)/# of columns e.g. (1-(4+1)*0.024)/4 = 0.22
% Set twocolwid to be (2*onecolwid)+sepwid = 0.464
% Set threecolwid to be (3*onecolwid)+2*sepwid = 0.708

\newlength{\sepwid}
\newlength{\onecolwid}
\newlength{\twocolwid}
\newlength{\threecolwid}
\setlength{\paperwidth}{47.8in} % A0 width: 46.8in
\setlength{\paperheight}{33.1in} % A0 height: 33.1in
\setlength{\sepwid}{0.020\paperwidth} % Separation width (white space) between columns
\setlength{\onecolwid}{0.30\paperwidth} % Width of one column
\setlength{\twocolwid}{0.6\paperwidth} % Width of two columns
\setlength{\threecolwid}{0.9\paperwidth} % Width of three columns
\setlength{\topmargin}{-0.5in} % Reduce the top margin size
%-----------------------------------------------------------
\usepackage{booktabs} % Top and bottom rules for tables
\usepackage{verbatim}
\usepackage{graphicx}
\usepackage{amsmath, amssymb, amsthm}
\usepackage{mathtools}
\usepackage{tabularx}
\usepackage{etex}
\usepackage{exscale}
\usepackage{multirow}
\usepackage{multicol}
\usepackage{mathtools}
\usepackage{fancyvrb}
\usepackage{algorithm}
\usepackage{algpseudocode}
\usepackage{float}
\usepackage{listings}
\usepackage{algorithm}
\usepackage{algorithmic}
\usepackage{times}
\usepackage{subfigure}

\setbeamerfont{itemize/enumerate subbody}{size=\normalsize} %to set the body size
\setbeamertemplate{itemize subitem}{\normalsize\raise1.25pt\hbox{\donotcoloroutermaths$\blacktriangleright$}}  %to set the symbol size
%\floatname{algorithm}{Procedure}
\renewcommand{\algorithmicrequire}{\textbf{Input:}}
\renewcommand{\algorithmicensure}{\textbf{Output:}}
\newcommand{\abs}[1]{\lvert#1\rvert}
\newcommand{\norm}[1]{\lVert#1\rVert}
\newcommand{\RR}{\mathbb{R}}
\newcommand{\CC}{\mathbb{C}}
\newcommand{\Nat}{\mathbb{N}}
\newcommand{\br}[1]{\{#1\}}
\renewcommand{\lstlistingname}{Code}
\DeclareMathOperator*{\argmin}{arg\,min}
\DeclareMathOperator*{\argmax}{arg\,max}

% Add the Logo Of CMU to the title
\setbeamertemplate{headline}{
 \leavevmode
  \begin{columns}
   \begin{column}{.1\linewidth}
    \includegraphics[width=1.1\linewidth]{CMUlogo.pdf}
   \end{column}
   \begin{column}{.6\linewidth}
    \centering
    \usebeamercolor{title in headline}{\color{CMURed}\Huge{\textbf{\inserttitle}}\\[0.5ex]}
    \usebeamercolor{author in headline}{\color{fg}\Large{\insertauthor}\\[0.2ex]}
    \usebeamercolor{institute in headline}{\color{fg}\Large{\insertinstitute}\\[0.5ex]}
    \vskip1cm
   \end{column}
   \begin{column}{.15\linewidth}
    \includegraphics[width=1.0\linewidth]{mllogo.jpg}
   \end{column}
  \end{columns}
 \hspace{0.5in}\begin{beamercolorbox}[wd=47in,colsep=0.15cm]{cboxb}\end{beamercolorbox}
}

\setbeamertemplate{footline}{%
  \leavevmode%
  \hspace{0.5in}\begin{beamercolorbox}[wd=47in,colsep=0.15cm]{cboxb}\end{beamercolorbox}
  \hbox{%
    \begin{beamercolorbox}[wd=.333333\paperwidth,ht=0.6in,dp=1.0in,left]{author in head/foot}%
      \usebeamerfont{author in head/foot}\color{CMURed}\Large{\hspace*{0.8in}NIPS-2017}\hspace*{2em}
    \end{beamercolorbox}%
    \begin{beamercolorbox}[wd=.333333\paperwidth,ht=0.6in,dp=1.0in,center]{title in head/foot}%
      \usebeamerfont{title in head/foot}\color{CMURed}\Large{December 6, 2017}\hspace*{2em}
    \end{beamercolorbox}%
    \begin{beamercolorbox}[wd=.333333\paperwidth,ht=0.6in,dp=1.0in,right]{date in head/foot}%
      \usebeamerfont{date in head/foot}\color{CMURed}\Large{han.zhao@cs.cmu.edu}\hspace*{2em}
    \end{beamercolorbox}
  }%
  \vskip0pt%
}
%----------------------------------------------------------------------------------------
%	TITLE SECTION 
%----------------------------------------------------------------------------------------

\title{Linear Time Computation of Moments in Sum-Product Networks}
\author{Han Zhao and Geoff Gordon}
\institute{Machine Learning Department, Carnegie Mellon University} 

%----------------------------------------------------------------------------------------

\begin{document}

\addtobeamertemplate{block end}{}{\vspace*{1ex}} % White space under blocks
\addtobeamertemplate{block alerted end}{}{\vspace*{1ex}} % White space under highlighted (alert) blocks

\setlength{\belowcaptionskip}{1ex} % White space under figures
\setlength\belowdisplayshortskip{1ex} % White space under equations

\begin{frame}[t] % The whole poster is enclosed in one beamer frame

\begin{columns}[t] % The whole poster consists of three major columns, the second of which is split into two columns twice - the [t] option aligns each column's content to the top

\begin{column}{\sepwid}\end{column} % Empty spacer column

\begin{column}{\onecolwid} % The first column
\vspace*{-0.5in}
\begin{block}{Background}
\textcolor{CMURed}{\textbf{Sum-Product Networks (SPNs)}:}
\begin{itemize}
    \item   Rooted directed acyclic graph of univariate distributions, sum nodes and product nodes.
    \item   We focus on discrete SPNs, but the proposed algorithms work for continuous ones as well.
\end{itemize}
\includegraphics[width=\linewidth]{spntrees.pdf}

Recursive computation of the network: 
\begin{equation*}
V_k(\mathbf{x};\mathbf{w}) = 
\begin{cases}
p(X_i = \mathbf{x}_i) & \text{if $k$ is a leaf node over $X_i$} \\
\prod_{j \in \text{Ch}(k)} V_j(\mathbf{x};\mathbf{w}) & \text{if $k$ is a product node} \\
\sum_{j\in \text{Ch}(k)} w_{k,j} V_j(\mathbf{x};\mathbf{w}) & \text{if $k$ is a sum node}
\end{cases} 
\end{equation*}
\textcolor{CMURed}{\textbf{Scope:}} The set of variables that have univariate distributions among the node's descendants. \\
\textcolor{CMURed}{\textbf{Complete:}} An SPN is \emph{complete} iff each sum node has children with the same scope.\\
\textcolor{CMURed}{\textbf{Decomposable:}} An SPN is \emph{decomposable} iff for every product node $v$, scope($v_i$) $\bigcap$ scope($v_j$) $=\varnothing$ where $v_i, v_j\in \text{Ch}(v), i\neq j$. 
\end{block}

\begin{block}{Summary}
\begin{itemize}
    \item   So far the best algorithm for moment computation scales quadratically in the size of Sum-product Networks (SPNs).
    \item   We propose an optimal linear-time algorithm that works even when the SPN is a general directed acyclic graph (DAG).
      \begin{itemize}
        \item   A linear time reduction from the moment computation problem to a joint inference problem.
        \item   An efficient procedure for polynomial evaluation by differentiation without expanding the network.
        \item   A dynamic programming method to reduce the computation of the moments of all the edges from quadratic to linear.
      \end{itemize}
    \item   Applications: 
      \begin{itemize}
        \item   Online moment matching
        \item   Assumed density filtering
      \end{itemize}
\end{itemize}
\end{block}
\end{column} % End of the first column

\begin{column}{\sepwid}\end{column} % Empty spacer column

\begin{column}{\onecolwid} 
\vspace*{-0.5in}
\begin{block}{Linear Time Exact Moment Computation}
\textcolor{CMURed}{\textbf{Exact Posterior Has Exponentially Many Modes:}}\\
Every complete and decomposable SPN $\mathcal{S}$ can be factorized into a sum of $\Omega(2^{H(\mathcal{S})})$ induced trees (sub-graphs), where each tree corresponds to a product of univariate distributions. $H(\mathcal{S}) = $ height of $\mathcal{S}$:
\begin{equation*}
p(\mathbf{w}\mid\mathbf{x}) = \frac{1}{Z_{\mathbf{x}}}\sum_{t=1}^{\tau}\prod_{k=1}^m\text{Dir}(w_k;\alpha_k)\prod_{(k, j)\in\mathcal{T}_{tE}}w_{k,j}\prod_{i=1}^n p_t(x_i)
\end{equation*}

$f$-moment on each edge in the network:
\begin{align*}
M_p(f(w_{k,j})) &= \int_{\mathbf{w}}f(w_{k,j})p(\mathbf{w}\mid\mathbf{x})~d\mathbf{w} \\
&= \frac{1}{Z_{\mathbf{x}}}\sum_{t=1}^{\tau} c_t\int_{\mathbf{w}}p_0(\mathbf{w})f(w_{k,j})\prod_{(k',j')\in\mathcal{T}_{tE}}w_{k',j'}~d\mathbf{w} 
\end{align*}
\begin{itemize}
  \item   Naive computation: $\Omega(|\mathcal{S}|\cdot 2^{H(\mathcal{S})})$
  \item   Recursive algorithm for trees: $O(|\mathcal{S}|)$, but $O(|\mathcal{S}|^2)$ for DAGs (Rashwan et al., 2016)
\end{itemize}

\textcolor{CMURed}{\textbf{Linear Time Reduction to Joint Inference:}}

For each edge $(k,j)$, partition all the trees into two sets:
\begin{itemize}
  \item  $\mathcal{T}_T = \{\mathcal{T}_t: t\in[\tau], (k, j)\in \mathcal{T}_t\}$, trees containing the edge.
  \item  $\mathcal{T}_F = \{\mathcal{T}_t: t\in[\tau], (k, j)\not\in \mathcal{T}_t\}$, trees not containing the edge.
\end{itemize}

$M_p(f(w_{k,j})) =$
\begin{equation}
\left(\frac{1}{Z_{\mathbf{x}}}\sum_{\mathcal{T}_t\in\mathcal{T}_F} c_t u_t\right) M_{p_{0,k}}(f(w_{k,j})) + \left(\frac{1}{Z_{\mathbf{x}}}\sum_{\mathcal{T}_t\in\mathcal{T}_T} c_t u_t\right)M_{p'_{0,k}}(f(w_{k,j}))
\end{equation}

\textcolor{CMURed}{\textbf{Efficient Polynomial Evaluation by Differentiation: }}

\textcolor{CMURed}{\textbf{Key observation 1: }}$\sum_{t\in \mathcal{T}_T}^{\tau}c_tu_t + \sum_{t\in \mathcal{T}_F}^{\tau}c_tu_t$ can be computed in $O(|\mathcal{S}|)$ time and space in a bottom-up evaluation of $\mathcal{S}$, by re-defining the edge weights of the network.

\textcolor{CMURed}{\textbf{Key observation 2: }}$\sum_{\mathcal{T}_t\in\mathcal{T}_T} c_t u_t = w_{k,j}\left(\partial\sum_{t=1}^{\tau}c_tu_t/\partial w_{k,j}\right)$, and it can be computed in $O(|\mathcal{S}|)$ time and space in a top-down differentiation of $\mathcal{S}$, by the multilinear nature of the network polynomial.

\textcolor{CMURed}{\textbf{Example: }}
\begin{align*}
\quad g(x_1, x_2, x_3) &= 4x_1x_2 + 3x_2x_3 + 5x_1x_3\\
\Rightarrow\quad  x_1\frac{\partial}{\partial x_1}g(x_1, x_2, x_3) &= 4{\color{blue}{x_1}}x_2 + \xcancel{3x_2x_3} + 5{\color{blue}{x_1}}x_3 
\end{align*}
\end{block}
\end{column} 

\begin{column}{\sepwid}\end{column} % Empty spacer column
\begin{column}{\onecolwid} % The third column
\vspace*{-0.5in}
\begin{block}{Dynamic Programming}
For each edge $(k, j)$, moment can be computed in $O(|\mathcal{S}|)$ time and space. Naive computation leads to $O(|\mathcal{S}|^2)$ for all edges.

\begin{columns}
\begin{column}{0.4\linewidth}
\includegraphics[width=\linewidth]{update.pdf}
\end{column}
~
\begin{column}{0.6\linewidth}
$$D_k(\mathbf{x};\mathbf{w}) = \frac{\partial V_{\text{root}}(\mathbf{x};\mathbf{w})}{\partial V_k(\mathbf{x};\mathbf{w})}$$
\end{column}
\end{columns}
Sufficient statistics for each edge contains three terms:
\begin{itemize} 
  \item     Forward value $V_j$ at the child.
  \item     Backward value $D_k$ at the parent.
  \item     Edge weight $w_{k,j}$.
\end{itemize}
\textcolor{CMURed}{\textbf{Quadratic to Linear:}} For all edges $(k, j)$, moments can be computed in $O(|\mathcal{S}|)$ time, using only two more copies of the network. 
\end{block}

\begin{block}{Applications}
Used as subroutines to develop linear time algorithms for Bayesian moment matching (BMM) and assumed density filters (ADF). 

Runtime (log-seconds) over 20 real-world data sets.
\begin{figure}
\vspace*{-0.3in}
\centering
\includegraphics[width=\linewidth]{runtime.pdf}
\end{figure}
\end{block}

\end{column} % End of the third column

\end{columns} % End of all the columns in the poster

\end{frame} % End of the enclosing frame

\end{document}
